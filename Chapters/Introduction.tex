% Introduction

\chapter{Introduction} % Main chapter title

\label{Introduction} % For referencing the chapter elsewhere, use \ref{Chapter1} 

Since the muon was discovered in 1936~\cite{cloudchamber} its properties have been of great interest to the particle physics community. This includes how its behaviour differs from the electron and the origin of its larger mass. The answer to these questions could come from its anomalous magnetic moment. This refers to any deviation from Dirac theory where the gyromagnetic ratio g=2. This was a surprise result when first measured for the electron, which determined that the magnetic moment of the electron was larger than thought, giving g = 2(1 + a) with a defined as the anomalous magnetic moment. This anomaly is understood to originate from quantum fluctuations of the electromagnetic field around the particle. The measurement of this value with increasing precision was the main motivation for the advancement of Quantum Electrodynamics \cite{Reference10}. The drive to measure the magnetic moments of other particles has continued since. 

Currently the electron's anomalous magnetic moment is the most precise test of the Standard Model (SM) to date. This measurement agrees with the SM at the part per trillion (ppt) level \cite{electronmagmom}. Heavier leptons enable a probe into other SM regions as the sensitivity of charged lepton anomalous magnetic moments to Beyond the Standard Model (BSM) physics is proportional to $m^2$. This means that the muon is 40,000 times more sensitive than the electron. Typically, greater mass provides enhanced sensitivity to high-energy-scale physics. The tau lepton although much heavier still, has too short a lifetime to be viable for experiments. The measurement of the muon amomalous magnetic moment is an important probe into new physics as it can be measured and calculated to a very high precision. Therefore the muon is the perfect candidate to look for BSM contributions. This has been the driving force for numerous experiments to increase the precise of a muon anomalous magnet moment measurement. 

The discovery of parity violating weak decays enabled the design of experiments to measurement the muon anaomalous magnetic moment. These decays allowed the production of polarised muon beams, where the muons spin and momenta are parallel and the decay electrons emission being aligned to the muons spin. This allows the muons spin to be determined from the measurement of its decay electron. This process was used in all muon anamolous magnetic moments measurements starting with the CERN experiments. These consisted of three experiment running between 1958 to 1976 which each increased the precision of the measurement. A further experiment carried out at the Brookhaven National Laboratory (BNL) the E821 experiment, which finished data taking in 2001 and achieved the most precise measurement to date with a precision of 0.54 parts per billion (ppb). 

The BNL experiment showed a discrepancy between the measured and theoretically predicted SM value. This discrepancy is currently at the 3.3$\sigma$-3.6$\sigma$ level depending on the particular theoretical model used. This falls short of the 5$\sigma$ discovery threshold normally required in the field of particle physics. Meaning the discrepancy could still be explained away as a statistical fluctuation. The discrepancy has lead to several BSM theory candidates being postulated to explain this. The lack of new physics found so far at the LHC means this measurement is an interesting alternative in the search of finding new physics.

The muon E989 g-2 experiment at Fermilab aims to increase the precision on the $\omega_{a}$ measurement by a factor of 4 from the BNL experiment in order to prove or refute this discrepancy. The Fermilab experiment has a goal of measuring the anomalous magnetic moment to a precision of 0.14 parts per million (ppm). It is based on the BNL experiment, reusing the BNL 1.45 T 14 m diameter storage ring, which was transported to Fermilab in 2013. With the Fermilab experiment having added improvements to attain the precision required for this measurement. The three storage ring subsystems: the kicker, the quadrupoles and the inflector which control the storage of the muon beam have been improved or replaced with the aim of improving the storage of muons by a factor of 2 or more. The purity of the muon beam delivered to the storage ring will also be greatly improved. In the BNL experiment, the beamline between the pion production target and the storage ring was less than the decay length of a 3.11 GeV/c pion. This meant there was a contamination of undecayed pions along with muons in the storage ring. The muon accelerator complex in Fermilab provides a higher intensity muon beam with much less pion contamination. The Fermilab muon campus produces a pure muon beam at a fill frequency rate $sim{3}$ times more than BNL. The experiment provides improvements to the detectors and the installation of two newly designed tracking station detectors. The experiment will also increase the uniformity and measurement of the storage ring magnetic field with large scale shimming processes and additional measurement equipment.

The tracking detectors which are a major part of this thesis consist of two stations each containing 8 tracking modules. These contain 128 straws filled with an Ar:Ethane gas mixture each containing a sense wire. The tracking detectors are used to measure the muon beam profile by extrapolating the trajectory of the decay positrons back to the point of muon decay. The tracking detectors can also measure beam dynamic properties and how these effect the $\omega_{a}$ measurement in order to improve the accuracy of the measurement. The trackers also enable cross checks between it and the calorimeter data.

The fist physics run started in November 2017 with several months of fine tuning the beam injection and storage prior to data taking beginning. Two physics runs have so far been completed, achieving just over 4 times the BNL statistics. A third data run is due to begin in October 2019 and aims to reach 10 times the BNL statistics.

The straw tracking detector operating principles, construction and quality assurance testing encompasses a major part of this thesis. The tracking detectors were built and tested in the University of Liverpool cleanrooms. I was involved heavily in almost every step of the construction and testing of the straw tracking detectors which took up the majority of the first 2 1/2 years of my PhD. I was also in charge of the metrology of the machined pieces of the trackers. This involved a metrology survey of each piece, the decision on which piece is passed or the changes required and using this data to match together pieces for each module. Using tracking data, I carried out a beam dynamics study. This was to characterise the behaviour of the vertical components of the muon beam throughout the fill. Looking at this behaviour variations in the expected frequencies were observed. This variation needed to be characterised more precisely so that it could be included in the $\omega_{a}$ fit function, and not distort the extracted value of $\omega_{a}$.


