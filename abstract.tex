The measurement of the anomalous magnetic moment of electrons and muons has been an important test of the Standard Model (SM) of particle physics over many decades. This is because it can be measured experimentally and calculated theoretically to a high precision. In particular the anomalous magnetic moment of the muon, $a_{\mu}$, is an ideal candidate for the search of new physics due to the combination of its large mass and relatively long lifetime.

The current world's most precise value of $a_{\mu}$ was measured by the E821 experiment at the Brookhaven National laboratory (BNL). This achieved a precision of 540 ppb and caused great interested by showing a $\sim 3.5\sigma$ deviation from the SM value. This motivated a new experiment: the E989 muon g-2 experiment at the Fermi National Accelerator Laboratory to confirm or reject the current discrepancy. This experiment aims to gather a data sample 21 times larger than the BNL experiment and to improve the determination of the systematic certainties by a factor of three to achieve a fourfold increase in precision to 140 ppb. With this improvement in precision, and if the $a_{\mu}$ value remains unchanged, it would establish evidence for previously unknown physical interactions by giving a $\sim$ 7$\sigma$ discrepancy between the experimental measurement and the SM calculation.

The Fermilab experiment is based on the BNL experiment and reuses the experiment’s storage ring magnet. Improved experimental equipment has also been introduced to reduce the systematic uncertainty on the $a_{\mu}$ measurement. One such improvement is the addition of two straw tracking stations which measure the trajectory of the positrons emitted from the (positive) muon decays and allow a detailed study of the spatial and temporal motion of the beam to be undertaken along with critical cross-checks of the the calorimeter data.

This thesis describes in detail the design, construction and testing of the tracking detectors which were built at the University of Liverpool. A detailed study of the vertical motion of the beam is also presented. The study provides an important correction that must be applied to the data before $a_\mu$  can be determined. 
